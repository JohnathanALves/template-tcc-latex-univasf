%--------------------------------------------------------------------------------------
% Este arquivo contém a sua introdução, objetivos e organização do trabalho
%--------------------------------------------------------------------------------------
\chapter{Introdução}

O desafio de levar educação e formação profissional a lugares remotos, onde,
dificilmente, a formação presencial tradicional conseguiria alcançar de maneira
efetiva, é uma das principais bandeiras da Educação a Distância (EAD).
Entretanto, outros desafios surgem em decorrência da expansão da modalidade:
como garantir a qualidade dessa formação? como transpor modelos de educação
presencial para a distância? como atuar de maneira a prevenir e reduzir os altos
índices de evasão, ainda verificados na modalidade? São questões como essas que
devem ser respondidas a partir do desenvolvimento de pesquisas nessa modalidade.
O uso de novas tecnologias e de novos processos podem contribuir com essas
pesquisas.

%-------------------suprimido temporariamente -------------------------------------------------------
% A partir do uso de algoritmos de aprendizagem de máquina já consolidados na área
% de mineração de dados, pretende-se testar modelos preditivos da evasão de alunos
% na EAD, que foram testados em alunos e cursos de outra instituição, para
% avaliar a sua aplicabilidade nos cursos de graduação a distância
% ofertados pela Universidade Federal do Vale do São Francisco (UNIVASF).

%-------------------suprimido temporariamente \section{Justificativa}

Diversas iniciativas reforçam o crescimento da EAD e exigem uma atenção maior
nos aspectos importantes para a consolidação e manutenção das atividades dessa
modalidade nas instituições. Dentre esses aspectos, está a necessidade de
pesquisas, como forma de se agregar procedimentos validados cientificamente,
ferramentas de gestão mais eficientes e metodologias inovadoras, capazes de
superar grandes desafios impostos pela EAD.

O avanço da modalidade de EAD requer o desenvolvimento de recursos que permitam
o acompanhamento de cursos oferecidos em um ambiente virtual de aprendizagem.
Esses recursos podem ser obtidos a partir de metodologias de análise que
envolvem o conhecimento das estratégias pedagógicas dos cursos EAD, o
levantamento das necessidades apontadas pelos profissionais que atuam na área e
na elicitação dos requisitos para implementação de ferramentas de visualização
de dados de diversas atividades dentro de um contexto educacional.
\cite{ramos2016abordagem}

Com a expansão do EAD de maneira responsável e planejada, com infraestrutura
compatível e recursos humanos qualificados, será possível a oferta de novos
cursos pelas instituições, disseminando conhecimento e possibilitando mais
oportunidades para o desenvolvimento regional.

Para \citeonline{ramos2016abordagem}, os altos índices de evasão dos alunos em
cursos EAD representam um grande desafio para todos os que atuam na modalidade.
Além desses índices estarem em níveis elevados, observou-se também que estão em
crescimento. Com isso há uma necessidade contínua de desenvolvimento de
pesquisas que apontem caminhos, métodos e ferramentas que os auxiliem a
enfrentar melhor esse problema. O uso de técnicas estatísticas e de mineração de
dados, em conjunto com teorias consolidadas na modalidade, pode fundamentar
modelos eficientes de detecção precoce do risco de evasão pelos alunos.

No estudo apresentado por \citeonline{ramos2016abordagem}, foram desenvolvidos,
testados e validados, modelos preditivos da evasão de estudantes de graduação em
cursos ofertados na modalidade EAD, tomando como base as variáveis que compõem
cada um dos construtos da Teoria da Distância Transacional
\cite{moore2008teoria}. Essa pesquisa ocorreu a partir dos dados de cursos de
licenciatura em Biologia e Pedagogia, ambos ofertados por EAD, na Universidade
de Pernambuco (UPE).

A citada pesquisa testou cinco algoritmos de classificação para definição dos
modelos preditivos: Árvore de Decisão, Máquina de Vetor de Suporte (SVM, do inglês, \textit{Support Vector Machine}), Rede
Neural Artificial, K-Vizinhos Mais Próximos (KNN, do inglês, \textit{K-nearest
Neighbors}) e Regressão Logística, sendo este último o que apresentou resultados
mais relevantes, embora os demais não ficaram muito distantes, nas métricas
analisadas.

A partir dessa referência, este estudo foi desenvolvido no sentido de verificar
se o mesmo conjunto de variáveis usadas e os algoritmos de classificação
aplicados, podem também ser replicados e validados em outro cenário educacional.
Desta vez nos cursos de graduação em Administração Pública e na Licenciatura em
Biologia, ofertados também por EAD, mas pela Universidade Federal do Vale do São
Francisco (UNIVASF).

Algumas adpatações no processo de replicação do estudo foram necessários, tais
como: mudança de tecnologia, ajuste nos scripts de coleta de dados e redução de
cinco para três algoritmos de classificação. Essas alterações não alteraram os
objetivos do trabalho, apenas forneceram novas e adequadas condições para o seu
desenvolvimento.

Assim, a principal questão a ser esclarecida neste trabalho é se um conjunto de
variáveis representativas da Teoria da Distância Transacional (TDT) e os alguns
dos algoritmos classificadores, também podem ser usados em modelos preditivos de
evasão na EAD, em um cenário diferente do originalmente apresentado por
\citeonline{ramos2016abordagem}.

Espera-se com este trabalho contribuir para o fortalecimento da EAD, além de
fomentar a linha de pesquisa voltada para o estudo das tecnologias educacionais,
tão evidenciadas e diversificadas, a partir do uso cada vez maior das
tecnologias de informação e comunicação no processo de ensino/aprendizagem,
particularmente aquelas destinadas a reduzir os atuais índices de evasão
verificados na modalidade.

\section{Objetivos}

Esta pesquisa será desenvolvida com o propósito de atingir os seguintes
objetivos geral e específicos:

\subsection{Objetivo geral}

Avaliar se um conjunto de modelos de predição utilizados em outro cenário
de EAD, pode também ser usado para prever quais alunos têm tendência a evasão em
cursos nessa modalidade na UNIVASF, mantendo os resultados em níveis
satisfatórios, comparados aos originais.

\subsection{Objetivos específicos}
\begin{itemize}
  \item Adaptar os modelos preditivos já desenvolvidos para uma outra ferramenta
  tecnológica;
  \item Aplicar os classificadores em bases de dados de cursos EAD da UNIVASF;
  \item Avaliar os resultados dos classificadores segundo métricas consolidadas.
\end{itemize}

\section{Organização do texto}

Esse trabalho está organizado em 5 capítulos. No primeiro capítulo apresenta-se
o projeto, uma contextualização sobre o problema abordado, assim como os
objetivos gerais e específicos.

No segundo capítulo, é realiada uma revisão sobre a TDT, MDE e Aprendizagem
Supervisionada, com objetivo de promover um maior detalhamento sobre os
conceitos utilizados ao longo do texto. Também neste capítulo são apresentados
resumos de trabalhos relacionados com esta pesquisa.

O terceiro capítulo explora os detalhes da caracterização da pesquisa e a
metodologia aplicada, Descoberta de Conhecimento em Bases de Dados (KDD, do
inglês \textit{Knowledge Discovery in Databases}).

O quarto capítulo contém o cronograma de atividades para a disciplina Trabalho
de Conclusão de Curso II (TCC II).

E por fim, o quinto capítulo contém as considerações finais, os resultados
esperados e a contribuição da pesquisa.

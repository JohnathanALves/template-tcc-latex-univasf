\chapter{Resultados}

Nesse capítulo são detalhados os resultados obtidos em cada faze do processo de
KDD aplicado nesse trabalho. Ele está nas mesmas seções do capítulo anterior,
pois cada faze do KDD contribui para o resultado final do processo e gera seus
próprios artefatos.

\section{Entrada de dados}

\section{Pré-processamento}

O Quadro \ref{reducedDataTables} apresenta e resume essas tabelas auxiliares.

\begin{quadro}[!htb]
  \centering
  \caption{Descrição das tabelas criadas apenas com dados dos cursos selecionados}
  \label{reducedDataTables}
  \begin{tabular}{|l|l|}
    \hline
    \multicolumn{1}{|c|}{\textbf{Tabela}} & \multicolumn{1}{c|}{\textbf{Descrição}} \\ \hline
    \textit{adm} & \begin{tabular}[c]{@{}l@{}}Tabela base agrupando dados de cada aluno do curso \\ Bacharelado em Administração Pública em cada \\ disciplina em que ele cursou.\end{tabular} \\ \hline
    \textit{adm\_base\_log\_reduzido} & \begin{tabular}[c]{@{}l@{}}Tabela com registros de log dos alunos e cursos do curso\\ Bacharelado em Administração Pública.\end{tabular} \\ \hline
    \textit{adm\_disciplinas} & \begin{tabular}[c]{@{}l@{}}Registra o identificador das disciplinas do curso de\\ Bacharelado em Administração Pública, data de início \\ e data de fim.\end{tabular} \\ \hline
    \textit{adm\_id\_alunos} & \begin{tabular}[c]{@{}l@{}}Registra o identificador dos alunos matriculados no \\ curso de Bacharelado em Administração Pública.\end{tabular} \\ \hline
    \textit{lic\_ped} & \begin{tabular}[c]{@{}l@{}}Tabela base agrupando dados de cada aluno do curso \\ Licenciatura em Pedagogia em cada disciplina\\ em que ele cursou.\end{tabular} \\ \hline
    \textit{lic\_ped\_base\_log\_reduzido} & \begin{tabular}[c]{@{}l@{}}Tabela com registros de log dos alunos e cursos do curso\\ Licenciatura em Pedagogia.\end{tabular} \\ \hline
    \textit{lic\_ped\_disciplinas} & \begin{tabular}[c]{@{}l@{}}Registra o identificador das disciplinas do curso de\\ Licenciatura  em Pedagogia, data de início e data de fim.\end{tabular} \\ \hline
    \textit{lic\_ped\_id\_alunos} & \begin{tabular}[c]{@{}l@{}}Registra o identificador dos alunos matriculados no\\  curso de Licenciatura em Pedagogia.\end{tabular} \\ \hline
  \end{tabular}
  \Ididthis
\end{quadro}

O Quadro \ref{variablesDescritionTable} descreve essas
variáveis e relaciona as mesmas com os construtos da TDT e seus identificadores.

\begin{quadro}[!htb]
  \centering
  \caption{Lista das variáveis e respectivos construtos e seus identificadores}
  \label{variablesDescritionTable}
  \begin{tabular}{|l|l|l|}
    \hline
    \multicolumn{1}{|c|}{\textbf{Identificador}} & \multicolumn{1}{c|}{\textbf{Variáveis}} & \multicolumn{1}{c|}{\textbf{Construto}} \\ \hline
    VAR01 & \begin{tabular}[c]{@{}l@{}}Quantidade geral de postagens do aluno em fóruns,\\ por disciplina.\end{tabular} & Diálogo \\ \hline
    VAR02 & \begin{tabular}[c]{@{}l@{}}Quantidade geral de mensagens enviadas pelo\\ aluno dentro do ambiente, por semestre.\end{tabular} & Diálogo \\ \hline
    VAR03 & \begin{tabular}[c]{@{}l@{}}Quantidade geral de mensagens recebidas pelo\\ aluno dentro do ambiente, por semestre.\end{tabular} & Diálogo \\ \hline
    VAR04 & \begin{tabular}[c]{@{}l@{}}Quantidade geral de recursos disponibilizados pelo\\ professor (página web, vídeo, pdfs, entre outros)\\ por disciplina.\end{tabular} & Estrutura \\ \hline
    VAR05a & \begin{tabular}[c]{@{}l@{}}Quantidade de acessos do aluno ao ambiente por\\ turno (Manhã), por semestre.\end{tabular} & Autonomia \\ \hline
    VAR05b & \begin{tabular}[c]{@{}l@{}}Quantidade de acessos do aluno ao ambiente por\\ turno (Tarde), por semestre.\end{tabular} & Autonomia \\ \hline
    VAR05c & \begin{tabular}[c]{@{}l@{}}Quantidade de acessos do aluno ao ambiente por\\ turno (Noite), por semestre.\end{tabular} & Autonomia \\ \hline
    VAR06 & \begin{tabular}[c]{@{}l@{}}Quantidade de colegas diferentes para quem o\\ aluno enviou mensagens no ambiente, por\\ semestre.\end{tabular} & Diálogo \\ \hline
    VAR07 & \begin{tabular}[c]{@{}l@{}}Quantidade de acessos do aluno ao ambiente no\\ semestre.\end{tabular} & Autonomia \\ \hline
    VAR08 & \begin{tabular}[c]{@{}l@{}}Quantidade de mensagens enviadas pelo aluno\\ aos professores pelo ambiente, por semestre.\end{tabular} & Diálogo \\ \hline
    VAR09 & \begin{tabular}[c]{@{}l@{}}Quantidade de mensagens dos professores recebidas\\ pelo aluno no ambiente, por semestre.\end{tabular} & Diálogo \\ \hline
    VAR10 & \begin{tabular}[c]{@{}l@{}}Quantidade de mensagens de colegas recebidas pelo\\ aluno no ambiente, por semestre.\end{tabular} & Diálogo \\ \hline
    VAR11 & \begin{tabular}[c]{@{}l@{}}Quantidade de mensagens enviadas pelo aluno para\\ outros colegas no ambiente, por semestre.\end{tabular} & Diálogo \\ \hline
    VAR12 & \begin{tabular}[c]{@{}l@{}}Quantidade de atividades com prazos de resposta ou\\ envio definidos por professor, por disciplina.\end{tabular} & Estrutura \\ \hline
    VAR13 & \begin{tabular}[c]{@{}l@{}}Quantidade de acessos do aluno aos diferentes tipos\\ de atividades disponibilizadas (webquest, fórum,\\ quiz, entre outros), por disciplina.\end{tabular} & Autonomia \\ \hline
    VAR14 & \begin{tabular}[c]{@{}l@{}}Quantidade de fóruns de discussão disponibilizados\\ sobre os conteúdos por disciplina.\end{tabular} & Estrutura \\ \hline
    VAR15 & \begin{tabular}[c]{@{}l@{}}Quantidade de acessos do aluno aos fóruns, por\\ disciplina.\end{tabular} & Autonomia \\ \hline
  \end{tabular}
  \Otherguydidthis{ramos2016abordagem}
\end{quadro}

\section{Mineração de Dados}

\section{Pós-processamento}

\section{Conhecimento Obtido}

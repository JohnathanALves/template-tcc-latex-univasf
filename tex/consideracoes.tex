\chapter{Considerações Finais e Trabalhos Futuros}

Ao final desse estudo, conclui-se que o conjunto de variáveis selecionado pode
ser utilizado para alimentar algoritmos de classificação da evasão na EAD.

As ferramentas selecionadas atenderam aos requisitos e apresentaram resultados
semelhantes aos encontrados na literatura, o que valida seu uso em trabalhos
futuros.

A modalidade EAD ajuda a democratizar o ensino, levando-o às regiões de difícil
acesso aos professores ou dando a oportunidade ao estudante de criar sua própria
rotina de estudos. A evasão desta modalidade de ensino ainda é um grande
problema a ser resolvido, logo, existe a necessidade de pesquisa científica
nesta área.

Com o uso crescente de ferramentas de tecnologia da informação em EAD fica
evidente que o uso de aprendizagem de máquina pode ser utilizado para modelar e
prever os fenômenos que causam a evasão.

O fluxo de descoberta de conhecimento em bases de dados descrito na metodologia
deste trabalho foi utilizado como arcabouço para a aplicação dos algoritmos de
classificação que foram comparados segundo métricas consolidadas na literatura.
Os resultados obtidos foram satisfatórios, aproximando-se aos encontrados na
literatura.

Os resultados obtidos neste trabalho, apontam que o uso das variáveis obtidas
a partir dos contrutos da Teoria da Distância Transacional, podem ser usadas
em ferramentas ou modelos preditivos da evasão dos alunos na EAD na UNIVASF,
contemplando o objetivo geral do projeto.

Como trabalhos futuros, propõe-se a elaboração de um sistema de controle para
professores e gestores, que, integrado a um modelo de classificação, ajude a
executar alguma ação no curso antes que a evasão do aluno ocorra. Sinalizações
automáticas de alunos com risco de evasão podem ser implementadas, para alertar
professores e tutores sobre essa situação. Além disso, um estudo aprofundado
sobre relevância das variáveis preditoras, pode levar a uma redução da
dimensionalidade das variáveis, sem perda do poder preditivo dos modelos. Isso
pode impactar na perfomance do processo, pois com menos variáveis, uma menor
quantidade de consultas ao banco de dados e um menor tempo de processamento dos
modelos podem ser alcançados.

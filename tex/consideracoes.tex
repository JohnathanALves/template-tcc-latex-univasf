\chapter{Considerações Finais e Trabalhos Futuros}

A modalidade EAD ajuda a democratizar o ensino, levando-o às regiões de difícil
acesso aos professores ou dando a oportunidade ao estudante de criar sua própria
rotina de estudos. A evasão desta modalidade de ensino ainda é um grande
problema a ser resolvido, logo, existe a necessidade de pesquisa científica
nesta área.

Com o uso crescente de ferramentas de tecnologia da informação em EAD fica
evidente que o uso de aprendizagem de máquina pode ser utilizado para modelar e
prever os fenômenos que causam a evasão.

O fluxo de descoberta de conhecimento em bases de dados descrito na metodologia
deste trabalho foi utilizado como arcabouço para a aplicação dos algoritmos de
classificação que foram comparados segundo métricas consolidadas na literatura.
Os resultados obtidos foram satisfatórios, aproximando-se aos encontrados na
literatura.

Como trabalho futuro, propõe-se a elaboração de um sistema de controle para
professores e gestores, que, integrado a um modelo de classificação, ajude a
executar alguma ação no AVA antes que a evasão do aluno ocorra.
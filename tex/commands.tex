% -----------------------------------------------------------------------------
% Pacotes fundamentais
% -----------------------------------------------------------------------------
\usepackage{xcolor}
\newcommand\myworries[1]{\textcolor{red}{[#1]}}
\usepackage{lmodern}		% Usa a fonte Latin Modern (Serifada, tipo Times New Roman
%\usepackage{helvet}		% Usa a fonte Helvetica (Tipo Arial)
%\renewcommand{\familydefault}{\sfdefault} tira o serifado
\usepackage[T1]{fontenc}		% Selecao de codigos de fonte.
\usepackage[utf8]{inputenc}		% Codificacao do documento (conversão automática dos acentos)
\usepackage{indentfirst}		% Indenta o primeiro parágrafo de cada seção.
\usepackage{color}				% Controle das cores
\usepackage{tikz}				% Inclusão de gráficos
\usepackage{graphicx}			% Inclusão de gráficos
\usepackage{microtype} 			% para melhorias de justificação
% -----------------------------------------------------------------------------
% Pacotes adicionais, usados no anexo do modelo de folha de identificação
% -----------------------------------------------------------------------------
\usepackage{multicol}
\usepackage{multirow}
% -----------------------------------------------------------------------------
% Pacotes adicionais, usados apenas no âmbito do Modelo Canônico do abnteX2
% -----------------------------------------------------------------------------
\usepackage{lipsum}				% para geração de dummy text
% -----------------------------------------------------------------------------
% Pacotes de citações
% -----------------------------------------------------------------------------
\usepackage[brazilian,hyperpageref]{backref}	 % Paginas com as citações na bibliografia
\usepackage[alf,abnt-etal-list=3,abnt-etal-cite=2, abnt-emphasize=bf,abnt-repeated-author-omit=yes,abnt-etal-text=emph]{abntex2cite}	% Citações padrão ABNT
\usepackage{pdflscape}
\usepackage{footnote}
\usepackage{pdfpages}
\usepackage{caption}

% -----------------------------------------------------------------------------
% Pacotes adicionados por @leolleocomp
% -----------------------------------------------------------------------------
\usepackage{booktabs}
\usepackage{adjustbox}
\usepackage{subcaption}
\usepackage[labelfont=bf]{caption}
\usepackage{gensymb}
\usepackage{amsmath}
\usepackage{array}
% \usepackage{float}
\usepackage{xcolor,colortbl}
\usepackage{longtable}
\usepackage{scalefnt}
\usepackage{listings}			% inserir codigo fonte
\usepackage{morewrites} % necessário pois estamos screvendo muitos arquivos
\usepackage{enumitem}

% -----------------------------------------------------------------------------
% Pacotes adicionados por @Gabrielr2508
% -----------------------------------------------------------------------------
\usepackage{hyperref}

\usepackage{tocloft}
% -- permite a adição de células especiais em tabelas
\newcommand{\specialcell}[2][c]{%
  \begin{tabular}[#1]{@{}c@{}}#2\end{tabular}}

\newcounter{equationset}
\newcommand{\equationset}[1]{% \equationset{<caption>}
  \refstepcounter{equationset}% Step counter
  \noindent\makebox[\linewidth]{Equação ~\theequationset: #1}
 }

%-------------------------------------------------------------------------------
% Adequação dos títulos dos capitulos, seções, subseções às normas da Univasf
% Added by @Gabrielr2508
%-------------------------------------------------------------------------------
\renewcommand{\ABNTEXchapterfont}{\fontseries{b}}
\renewcommand{\ABNTEXchapterfontsize}{\normalsize}

\renewcommand{\ABNTEXsectionfont}{\fontseries{m}}
\renewcommand{\ABNTEXsectionfontsize}{\normalsize}

\renewcommand{\ABNTEXsubsectionfont}{\fontseries{b}}
\renewcommand{\ABNTEXsubsectionfontsize}{\normalsize}

\renewcommand{\ABNTEXsubsubsectionfont}{\fontseries{m}}
\renewcommand{\ABNTEXsubsubsectionfontsize}{\normalsize}

%-------------------------------------------------------------------------------
% CONFIGURAÇÕES DE PACOTES
% Configurações do pacote backref
%-------------------------------------------------------------------------------
% Usado sem a opção hyperpageref de backref
\renewcommand{\backrefpagesname}{Citado na(s) página(s):~}
% Texto padrão antes do número das páginas
\renewcommand{\backref}{}
% Define os textos da citação
\renewcommand*{\backrefalt}[4]{
  \ifcase #1 %
    %Nenhuma citação no texto.%
  \or
    Citado na página #2.%
  \else
    Citado #1 vezes nas páginas #2.%
  \fi}%

%-------------------------------------------------------------------------------
% Configurações de aparência do PDF final
%-------------------------------------------------------------------------------
% alterando o aspecto da cor azul
\definecolor{blue}{RGB}{41,5,195}

% informações do PDF
\makeatletter
\hypersetup{
       %pagebackref=true,
    pdftitle={\@title},
    pdfauthor={\@author},
      pdfsubject={\imprimirpreambulo},
      pdfcreator={LaTeX with abnTeX2},
    pdfkeywords={abnt}{latex}{abntex}{abntex2}{relatório técnico},
    colorlinks=true,			% false: boxed links; true: colored links
      linkcolor=black,				% color of internal links
      citecolor=black,				% color of links to bibliography
      filecolor=black,			% color of file links
    urlcolor=black,
    bookmarksdepth=4
}
\makeatother
% ---

% ---
% Espaçamentos entre linhas e parágrafos
% ---

% O tamanho do parágrafo é dado por:
\setlength{\parindent}{1.3cm}

% Controle do espaçamento entre um parágrafo e outro:
\setlength{\parskip}{0.2cm}  % tente também \onelineskip

%-------------------------------------------------------------------------------
% compila o indice
%-------------------------------------------------------------------------------
\makeindex
% ---

%-------------------------------------------------------------------------------
% Comando para inserir imagens de forma simples
%-------------------------------------------------------------------------------
\newcommand{\imagem}[4]
{% \imagem{x.x}{nomeimg}{titulo}{fonte}
  \begin{figure}[!htb]
    \caption{\label{img:#2}#3}
    \begin{center}
      \includegraphics[scale=#1]{img/#2}
    \end{center}
        \legend{\textbf{Fonte:} #4.}
  \end{figure}
}%

%-------------------------------------------------------------------------------
% Creio que esses comandos sejam para desenhar algo, aguardando explicações de @leolleocomp
%-------------------------------------------------------------------------------
\newcommand{\xx} {$\bigotimes$}
\newcommand{\oo} {$\bigcirc$}

%-------------------------------------------------------------------------------
% Biblioteca para códigos-fonte
%-------------------------------------------------------------------------------
\usepackage[newfloat=true]{minted}

%-------------------------------------------------------------------------------
% Caixas batutas - by @leolleocomp
%-------------------------------------------------------------------------------
\usepackage[most]{tcolorbox}
\tcbuselibrary{breakable}

\tcbuselibrary{minted}
\tcbset{listing engine=minted}

\definecolor{bg}{rgb}{0.95,0.95,0.95}

\SetupFloatingEnvironment{listing}{name=Código, listname=Lista de códigos}

%-------------------------------------------------------------------------------
% configuração do contador dos códigos-fonte - by @leolleocomp
% assim como as figuras, começa em 1
\newcounter{sourcecode}
%-------------------------------------------------------------------------------
%-------------------------------------------------------------------------------
% @leolleocomp
% stackoverflow code
% peguei da resposta abaixo
% https://stackoverflow.com/questions/24086366/change-latex-minted-listings-numbering-to-include-current-section?answertab=votes#tab-top
%-------------------------------------------------------------------------------
\makeatletter
\renewcommand*{\thelisting}{\thesourcecode}
\makeatother

%-------------------------------------------------------------------------------
% Peçam explicações a @leolleo
% WHO DID THIS?
%-------------------------------------------------------------------------------
\newcommand{\Ididthis}{
%	\legend{\textbf{Fonte:} O autor (\the\year).}
\legend{\textbf{Fonte:} Elaborado pelo autor.}
}

\newcommand{\Otherguydidthis}[1]{
  \legend{\textbf{Fonte:} \citeonline{#1}.}
}

%-------------------------------------------------------------------------------
% Comando para inserir códigos - by @leolleocomp
%-------------------------------------------------------------------------------
\newcommand{\sourcecode}[4]{
\begin{listing}[h]
  \refstepcounter{sourcecode}
  \caption{#1}
  \label{cmd:#2}
  \inputminted[linenos, bgcolor=bg, tabsize=4,breaklines]{#3}{codes/#4}
  \Ididthis
\end{listing}
}


% -----------------------------------------------------------------------------
% Pacotes adicionados por @ruanmed
% -----------------------------------------------------------------------------
\usepackage[binary-units=true]{siunitx}


%-------------------------------------------------------------------------------
% Comando para inserir códigos - by @ruanmed
%-------------------------------------------------------------------------------
\usepackage{caption}

\newenvironment{code}{\captionsetup{type=listing}}{}
\SetupFloatingEnvironment{listing}{name=Código}

\newcommand{\sourcecodenolist}[4]{
  \begin{code}
      \refstepcounter{sourcecode}
        \captionof{listing}{#1 }
        \label{code:#2}
        \inputminted[linenos, bgcolor=bg, tabsize=4,breaklines]{#3}{codes/#4}
        \Wedidthis
    \end{code}
}

\newcommand{\sourcecodeinline}[2]{
  \mintinline[linenos, bgcolor=bg, tabsize=4,breaklines]{#1}{#2}
}


% CONFIGURACAO DO SUMARIO
%-------------------------------------------------------------------------------
% Modifica o espaçamento no sumário
% Nao ha espacos, exceto para as entradas de capitulos
\setlength{\cftbeforeparagraphskip}{0pt}
\setlength{\cftbeforesubsectionskip}{0pt}
\setlength{\cftbeforesectionskip}{0pt}
\setlength{\cftbeforesubsubsectionskip}{0pt}
\setlength{\cftbeforechapterskip}{\onelineskip}

% Alteração da indentação dos itens do sumário
\cftsetindents{chapter}{0pt}{42pt}
\cftsetindents{section}{0pt}{42pt}
\cftsetindents{subsection}{0pt}{42pt}
\cftsetindents{subsubsection}{0pt}{42pt}

% Modifica a formatacao dos textos

% Secao Primaria (Chapter): Caixa alta, Negrito, tamanho 12
\makeatletter
\settocpreprocessor{chapter}{%
  \let\tempf@rtoc\f@rtoc%
  \def\f@rtoc{%
  \texorpdfstring{\bfseries\MakeTextUppercase{\tempf@rtoc}}{\tempf@rtoc}}%
}
\makeatother

% Novo list of (listings) para QUADROS
% retirado de https://github.com/abntex/abntex2/wiki/HowToCriarNovoAmbienteListing

\newcommand{\quadroname}{Quadro}
\newcommand{\listofquadrosname}{Lista de quadros}

\newfloat[chapter]{quadro}{loq}{\quadroname}
\newlistof{listofquadros}{loq}{\listofquadrosname}
\newlistentry{quadro}{loq}{0}

% configurações para atender às regras da ABNT
\setfloatadjustment{quadro}{\centering}
\counterwithout{quadro}{chapter}
\renewcommand{\cftquadroname}{\quadroname\space}
\renewcommand*{\cftquadroaftersnum}{\hfill--\hfill}

% Configuração de posicionamento padrão:
\setfloatlocations{quadro}{hbtp}

% Trocar o título da lista de ilustrações para lista de figuras
\addto\captionsbrazil{
  \renewcommand{\listfigurename}
    {Lista de figuras}
}
